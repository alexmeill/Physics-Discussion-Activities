\documentclass{article}

\usepackage[margin=0.5in]{geometry}

\title{Are galaxies electrically neutral?}
\author{Original Author: Nathan Butcher}
\date{}


\begin{document}
\maketitle
\thispagestyle{empty}

\textbf{Purpose of activity:} Start the class by using a simple principle (Coulomb's law) to reach a significant and interesting conclusion.

\textbf{Topic and objective:} This should be the first discussion section, ideally the first practice problem or example you do. Coulomb's law should be introduced at the very beginning of the class.

\textbf{Preparation:} Need to prepare 3-4 slides for the derivation or notes for a chalkboard derivation.

\textbf{Duration:} 10 minutes 

\textbf{Considerations and suggestions:} 

\hspace{14pt}

\textbf{Activity:}
\begin{enumerate}
\item Start by asking the class ``Are galaxies electrically neutral? How do you think we could determine that?" Responses should be interesting and varied.

\item State that we can compare the force due to gravity and the force due to electric charge. Write out Newton's law of universal gravitation (explain what the variables are) and ask how it compares to Coulomb's law. The two key points are the $1/r^2$ dependence and the product of masses vs. product of charges.

\item Have the class group up into groups of 2-4 students. Ask them to find the ratio of the Coulomb force to the gravitational force between two particles at some distance, with particle mass $m$ and charge $q$. Give the class the constants $G = 6.67 \times 10^{-11} \, m^3 \, kg^{-1} \, s^{-2}$ and $k_e = 9.0 \times 10^9 \, N \, m^2 \, C^{-2}$. Let the class work on this for about 2 minutes. The result is

\begin{equation}
	\frac{F_c}{F_g} = 1.35 \times 10^{20} \, \frac{kg^2}{C^2} \, \frac{q^2}{m^2}
\end{equation}

Emphasize that it is dimensionless, as we are taking the ratio of two quantities with the same units.

\item Now we will use protons as the particles. Plug in the mass $m_p = 1.67 \times 10^{-27} kg$ and the charge $q_p = 1.6 \times 10^{-19} C$ to show that for a pair of protons

\begin{equation}
	\frac{F_c}{F_g} = 1.24 \times 10^{36}
\end{equation}

\item Tell the class that gravitational forces can explain the motion of stars within a galaxy if we add in extra mass that is not visible (dark matter). In their same groups, ask them to think about how this observation plus the force ratio they calculated can help them determine if galaxies are electrically neutral. Give them a few minutes and then ask for responses.

\item The ratio of forces indicates that the Coulomb force between two protons is $1.24 \times 10^{36}$ times greater than the gravitational force. Therefore, if there was an excess of protons greater than 1 part in $1.24 \times 10^{36}$ the galaxy dynamics would not be so well described by gravity. This sets a limit on the charge imbalance for the galaxy. 

\item How would our limit change if we considered electrons instead of protons? Electrons have the same magnitude of charge with the opposite sign and a mass that is 1/2000 that of a proton. Ask them to spend 1 to 2 minutes individually coming up with an answer, and then pair up and discuss their answers for another minute. Finally, as for volunteers to share what their groups came up with.
\end{enumerate}

\textbf{Reflection:} For charged objects the Coulomb force is significantly stronger than gravity. This is why we will ignore gravity when solving problems about the forces charges experience throughout the quarter.

\hspace{14pt}



\end{document}